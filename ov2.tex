%%%%%%%%%%%%
% PREAMBLE %
%%%%%%%%%%%%

\documentclass[a4paper]{article}

\usepackage[utf8]{inputenc}
\usepackage[T1]{fontenc}
\usepackage[protrusion=true,expansion=true]{microtype}
\usepackage{parskip}
\usepackage{graphicx}
\usepackage{geometry}

\title{TIØ4258 -- Exercise 2 \\ Group 45}
\author{
    Anders Wold Eldhuset \and
    Péter Henrik Gombos \and
    Sindre Haneset Nygård \and
    Audun Skjervold \and
    Odd Magnus Trondrud
}
\date{\today}
\pagestyle{fancy}
\rhead{Group 45}
\lhead{TIØ4258 -- Exercise 2}
%%%%%%%%%%%%
% DOCUMENT %
%%%%%%%%%%%%

\begin{document}

    \maketitle
    \newpage
    \setcounter{secnumdepth}{2}

    \section{Innovation and Organizational Structure} % Oppgave 1

    \subsection*{a) Present two examples of employed divisional structure from
    two different cases and describe its advantages and disadvantages.}

    \subsubsection{NoRAS} On page three of the NoRAS case
    description\footnote{Part three, 1993-1994}, the company's structure is
    described as follows: ``The company is located in Namsos and consists
    of two production departments: one for microcontroller systems and one
    for the plastic casings for the control units. [...] In addition, the
    company has a small marketing-unit that visits customers.'' The company
    ``develops and sells custom RC-solutions for industry purposes'', which
    probably means neither of the production departments are anything like
    a product division. Or are they? It depends on the company's actual
    product, which is neither plastic casings nor microcontroller systems
    -- however both of these are products which the company develops and
    produces. For the sake of argument let's say the company has one product
    line: ``custom RC-solutions for industry purposes''. Now the company has
    one product line, one geographical location (Namsos), one market it caters
    to (``the industry'') and one set of support functions. Based on this, as
    well as what other information about the company that is provided by the
    case description up to and including Part 2, it would seem they employ
    a Product Team Structure, if we assume their Products are custom made
    RC-solutions and not the various components of such a solution.

    The Product Team Structure allows NoRAS to organize teams around each
    individual order they get, providing their customers with the fully
    customized thing (which is what they are all about).
    A downside to this is that this obsessive attentiveness to customization
    could limit their total throughput of products.

    \subsubsection{Nordvestbygg AS} When NORDVESTBYGG AS start expanding, they
    use a geographical structure. In the case description of the structure
    during the initial expansion, the structure are described as ``[...]
    focus on being represented in other cities as well as Ålesund. [...] all
    the acquired branches were working relatively autonomous, having their own
    management and making their own deals.'' This could be a good strategy
    for a construction company. The local branch will have good knowledge of
    the local laws and regulations and can adapt quickly to changes in the
    rules and regulations. However, this structure requires that each of the
    branches has a certain size. If they are too small you might end up with
    a lot of extra costs. For example the cost of a complete administration
    and management for each branch will cause a bigger overhead. Also the
    geographical structure might prevent knowledge exchange between the
    branches.

    After a while Nordvestbygg AS had expanded to almost 400 employees and
    the company was undergoing a reorganization. This is described as ``The
    employees were divided into 3 departments; project planning with ca 50
    people, technical services with about 100 people and construction work
    with about 200. The other 20 people worked in sales and administration.''
    This is product team structure where the different products are project
    planning, construction work, technical services, in addition to the sales
    and administration department. One of the main advantages is illustrated
    in the case when the company first tries to sell complete enterprises,
    products involving all the different departments. When this strategy fails
    they could quickly change their business strategy and sell the product of
    each department by itselves. Another advantage is that you can gather or
    the knowledge and resources associated with each field in one place. This
    reduces the overhead connected to duplicate resources and could enhance
    the quality of the project.

    \subsection*{b) For each case presented in 1-a, describe a different
    divisional structure the organization might choose, along with any
    advantages and disadvantages with it versus its current structure.}

    \subsubsection{NoRAS} It's hard to make a good case in favor of circa 1993
    NoRAS changing its corporate structure. One could even say they employ a
    functional structure, not a divisional one, at this point in time (circa
    1993). So let's cheat and look to the future (circa 2000), at which point
    they lay plans to expand into the North American oil drilling market,
    which is notably different from the North-Western European offshore oil
    market.

    NoRAS' product is still custom RC-solutions, yet the two markets they now
    operate in are so different that it would make sense for them to make the
    switch to Market Structure. This would allow them to better accommodate
    the different requirements for each of the markets they operate in.
    The switch would also increase horizontal differentiation within the
    company. It could also lead to a reduction in NoRAS' expertise within
    the North-Western European offshore market as their resources are split
    between the two markets, but this would be caused by them deciding to
    operate in two markets not that they change their divisional structure to
    accommodate this fact.

    Remaining with the Product Team Structure would keep NoRAS' focus on their
    product (custom RC-solutions, the key here is ``custom''), rather than the
    markets.

    \subsubsection{Nordvestbygg AS} When Nordvestbygg AS was bought by
    Norgesbygg AS in 1997 they were trying to enforce some very drastic
    changes. Perhaps in this case since the company was so prosperous, the
    new owners should not have tried to force the changes but instead they
    could have gone back to the original geographical structure. This way
    the Nordvestbygg branch could continue to operate more or less the same
    ways as before without being disturbed by the new owners. Implementing
    the knowledge exchange program mentioned in the case could also prevent
    the previously discussed disadvantages of the poor knowledge exchange
    that the geographical structure can result in. The fact that Nordvestbygg
    had almost 400 employees would also help with reducing the overhead of
    management. There could be made some cuts in the management, but the most
    essential parts would still be there.

    \section{Innovation and People} % Oppgave 2
    Job design is all about connecting various work tasks with fixed jobs.
    Good job design can increase motivation and improve performance.

    \subsection*{Describe the difference between job enlargement and job
    enrichment}

    In short terms: Job enlargement involves giving the employee additional
    tasks to perform at the same difficulty and responsibility level as the
    ones they already are assigned. Job enrichment involves giving them more
    responsibility and control over their work, such as planning their own
    schedules, deciding how to perform their work, checking their own work or
    learning new skills.

    Both these concepts (job enlargement and enrichment) are aimed at
    increasing intrinsic motivation in employees, which is to say the
    motivation that comes from the job itself, by enjoying the job. There are
    five key points that typically increase intrinsic motivation.


    \begin{description}
        \item[Skill variety] If the employee has to use a larger, varied skill
        set to complete his or her tasks, he is less likely to get bored.

        \item[Task identity] If the employee is involved in the work from
        beginning to end, as opposed to just being involved in a small
        fraction of the process, he is more likely to identify with the work,
        increasing motivation.

        \item[Task significance] If the employee feels the work he is doing is
        important to someone, he is likely to be more motivated.

        \item[Autonomy] If the employee has independence and power over how to
        perform his job, he may find it more enjoyable.

        \item[Feedback] If the employee receives feedback on the quality and
        progress while performing his job, it's easier to be motivated.
    \end{description}

    Job enrichment focuses on skill variety, task identity and most
    importantly, autonomy. Job enlargement does not have a clear focus on
    any of these points, but one could argue that it partly focuses on task
    identity, involving the employee in a larger part of the process. One
    could also say it involves skill variety, because the additional tasks may
    require additional skills, but only to a certain degree, as the tasks are
    supposed to be of a similar difficulty and responsibility.

    As mentioned, both methods aim to increase intrinsic motivation. However,
    job enlargement is a short-term measure, as additional simple tasks may
    get boring after a relatively short period of time. Job enrichment is
    a more long-term measure, as the points the method focuses on - mainly
    skill variety and autonomy - are aimed at keeping the employee interested
    and involved. Enrichment is not always applicable, however, as not all
    employees want more responsibility, and not all jobs have room for more
    responsibility, as it may decrease efficiency.

    \subsection*{Imagine that you are a management team tasked with
    redesigning one or more jobs from one of the cases. Suggest three measures
    designed to increase the motivation of the company's employees. (If
    applicable, describe any assumptions you make.)}

    Assume we are the management of Arendal Verft AS. Traditionally, a typical
    job would be a place at some part of an assembly line. For a supervisor,
    it might be quality control of his subordinates. In the case description,
    some measures were taken to increase motivation. Below I will explain what
    these measures were, as well as explain why and how they helped. Finally,
    I will mention additional measures that could have been taken.

    The management in the case decided to organize incoming orders as
    projects, where each project had a supervisor in the position of project
    manager, responsible for the economy of the project, as well as finding
    the right subordinates to work on it. The subordinates, or regular
    employees, would work full time on a project, while supervisors and
    specialists would be shared between projects if necessary. Organizing this
    way gives the employees a higher degree of task identity because they
    are involved in the entire project from beginning to end. Furthermore,
    supervisors have higher autonomy, because they as project managers have
    to select people to work on their project as well as see it through. They
    also have a higher degree of skill variety, because they have to take care
    of the economic aspect of the project as well. Arguably, this could also
    be said for the regular employees, as they might be included in tasks
    requiring different skills when they are a part of the entire project, as
    opposed to doing the same job on an assembly line for different projects.

    Another measure that was taken was training. Project managers were
    required to take on trainees, and would as a result be allowed to log
    more hours, thus receiving a larger paycheck. A larger paycheck is always
    motivating, but I have chosen to focus on the intrinsic motivations rather
    than the extrinsic - that is, outer - motivations, such as pay. Taking on
    trainees gives the project managers a higher degree of skill variety, as
    it calls for pedagogical skills. It also gives the supervisors a feeling
    of task significance, because they have an impact on the future of the
    trainees. As far as the trainees are concerned, I assume that the case
    describes new employees who need to learn the job of a regular employee.
    However, this trainee model could easily have been deployed as a way of
    further learning for the already established employees, training them to
    become supervisors and later potentially project managers, giving them the
    aforementioned boons to motivation.

    A final but important measure is the creation of a ``reference group''.
    This would be a group of representatives from the various ``classes''
    of employees that could gather periodically and discuss possible ways
    of improving the way their jobs are performed. This gives the group
    of representatives increased autonomy, as well as variation in skill
    usage. The group could also act as an interface between management and
    the remaining employees for concerns and requests, as well as feedback.
    This group would then be able to experience increased task significance,
    bringing the voice of the employees to management. Easily accessible
    feedback would also increase motivation.

    \section{Innovation} % Oppgave 3

    \subsection*{Provide one example of incremental and radical change, from
    the provided cases. Explain the more important differences between them.}

    Incremental innovation is the act of making changes to existing products
    so that you can get as much value as possible from the product, without
    having to change the business model of the company much, or to make big
    investments into something new. While not sounding as such an important
    part of a business, incremental innovation can be very important to keep
    the business current, and to keep attracting customers. [page 16-19]

    Radical innovation happens when a business comes up with a whole new
    idea, changing the business model, and how it's customers relate to the
    business. They can also be known as ``game changers,'' changing how the
    game of the industry works. While radical innovation can be good for
    a company, it might not be worth investing a lot in, as the nature of
    radical innovation is that nobody can know what the next ``big thing'' is.

    In the case study of Vest-Regnskap AB, there's an example of incremental
    innovation. As the company started, it used a computer system that was
    specially designed for VR. The administration started changing these out
    with new software, and added network systems, allowing remote personnel in
    the company.

    The company Tungesvik Stålsveis AS has had more than one radical
    innovation during its history. After almost fifty years of building wooden
    boats, they changed their production line to be mechanical, and started
    building boats of steel. Sixty years later again, the new administration
    added a new division creating constructions in stainless steel, while the
    traditional boat building division was closed a year later, turning the
    company into something completely different.

    \subsection*{Based on the information provided in the case description,
    show that one of the case-companies employ one of the four learning
    systems (value creation, improvement of processes, building competence,
    emerging strategies) and provide arguments for why they should
    change it (including both the advantages and disadvantages). (If
    applicable, describe any assumptions you make.)}

    The innovation of Bjørnsen \& Sønn Støperier relied heavily on the
    building of competence in their workforce. Requiring key personnel to
    undertake further education, and all employees to spend time working
    with their customers to assess how the quality of their product affects
    products ``down-stream'', led to a significant increase in their technical
    ability. This was a large and costly move, but probably a necessary one
    when faced with increasingly diverse orders. It also opened up a new market
    in that oil companies, whose demands were higher than those of the shipping
    industry, became possible customers. In short, increasing the quality of
    their staff allowed Bjørnsen \& Sønn to survive a change in the nature of
    their industry, as well as to enter a market with potentially higher
    profits.

    There are also elements of emerging strategies in the case of Bjørnsen
    \& Sønn, in that they pioneered the idea of purchasing training and
    maintenance along with a significant order of industrial equipment. While
    this added to the price, it also guaranteed that the tools would come to
    good use; presumably, the company that makes a tool will be able to teach
    the use of it to others, and a maintenance deal of five years removes the
    risk of costly equipment becoming useless shortly after its purchase.

    If I were to suggest a change, it would be to move even more in the
    direction of emerging strategies. Rather than just purchase the safety
    of maintenance and training from their suppliers, why not offer similar
    services themselves? With the highly educated workforce that they accrued,
    it would probably be possible to offer consulting services to the rather
    particular oil companies. Doing so would not only present another stream
    of income, but also create a certain level of lock-in on the part of the
    oil companies, which could provide long-term safety to Bjørnsen \& Sønn.

    The risk of moving from a strictly-production company to one that also
    offers consulting services, would be that they might find themselves
    ``spread too thin''; a talented factory worker is not necessarily as
    talented when it comes to customer relations, and hiring dedicated
    consulting staff would no longer piggyback on the general increase in
    competence at Bjørnsen \& Sønn. Still, it seems like an idea worth
    exploring.

\end{document}
